\RequirePackage{atbegshi}
\documentclass[compress,aspectratio=169]{beamer}\usepackage[]{graphicx}\usepackage[]{color}
%% maxwidth is the original width if it is less than linewidth
%% otherwise use linewidth (to make sure the graphics do not exceed the margin)
\makeatletter
\def\maxwidth{ %
  \ifdim\Gin@nat@width>\linewidth
    \linewidth
  \else
    \Gin@nat@width
  \fi
}
\makeatother

\definecolor{fgcolor}{rgb}{0.345, 0.345, 0.345}
\newcommand{\hlnum}[1]{\textcolor[rgb]{0.686,0.059,0.569}{#1}}%
\newcommand{\hlstr}[1]{\textcolor[rgb]{0.192,0.494,0.8}{#1}}%
\newcommand{\hlcom}[1]{\textcolor[rgb]{0.678,0.584,0.686}{\textit{#1}}}%
\newcommand{\hlopt}[1]{\textcolor[rgb]{0,0,0}{#1}}%
\newcommand{\hlstd}[1]{\textcolor[rgb]{0.345,0.345,0.345}{#1}}%
\newcommand{\hlkwa}[1]{\textcolor[rgb]{0.161,0.373,0.58}{\textbf{#1}}}%
\newcommand{\hlkwb}[1]{\textcolor[rgb]{0.69,0.353,0.396}{#1}}%
\newcommand{\hlkwc}[1]{\textcolor[rgb]{0.333,0.667,0.333}{#1}}%
\newcommand{\hlkwd}[1]{\textcolor[rgb]{0.737,0.353,0.396}{\textbf{#1}}}%
\let\hlipl\hlkwb

\usepackage{framed}
\makeatletter
\newenvironment{kframe}{%
 \def\at@end@of@kframe{}%
 \ifinner\ifhmode%
  \def\at@end@of@kframe{\end{minipage}}%
  \begin{minipage}{\columnwidth}%
 \fi\fi%
 \def\FrameCommand##1{\hskip\@totalleftmargin \hskip-\fboxsep
 \colorbox{shadecolor}{##1}\hskip-\fboxsep
     % There is no \\@totalrightmargin, so:
     \hskip-\linewidth \hskip-\@totalleftmargin \hskip\columnwidth}%
 \MakeFramed {\advance\hsize-\width
   \@totalleftmargin\z@ \linewidth\hsize
   \@setminipage}}%
 {\par\unskip\endMakeFramed%
 \at@end@of@kframe}
\makeatother

\definecolor{shadecolor}{rgb}{.97, .97, .97}
\definecolor{messagecolor}{rgb}{0, 0, 0}
\definecolor{warningcolor}{rgb}{1, 0, 1}
\definecolor{errorcolor}{rgb}{1, 0, 0}
\newenvironment{knitrout}{}{} % an empty environment to be redefined in TeX

\usepackage{alltt} % aspectratio=169
%\usepackage[svgnames]{xcolor}

%	%	%	%	%	%	%	%	%	%	%	%	%	%	%
% 						MY PACKAGES 
%	%	%	%	%	%	%	%	%	%	%	%	%	%	%
\usepackage{graphicx}				% Use pdf, png, jpg, or eps with pdflatex; use eps in DVI mode
\usepackage{dcolumn} % this pack is neccesary to build nicer columns with texreg--dont remove it.
\usepackage[export]{adjustbox}

\usepackage{amssymb}
\usepackage{amsmath}	
%\usepackage{tipx}
%\usepackage{tikz}
%\usetikzlibrary{arrows,shapes,decorations.pathmorphing,backgrounds,positioning,fit,petri}
\usepackage{rotating}
%\usepackage{scalerel} % for inline images
\usepackage{import}
%\usepackage{times}
\usepackage{array}
\usepackage{tabularx}
%\usepackage{booktabs}
%\usepackage{textcomp}
\usepackage{float}
%\usepackage{setspace} 			% \doublespacing \singlespacing \onehalfspacing	%doble espacio
%\label{x:y}													%ocupar para autoref.
%\autoref{x:y}												%ocupar para autoref.
%\usepackage{nopageno}			%desactivar para p<U+FFFD><U+FFFD><U+FFFD>ginas
\usepackage{pifont}
%\usepackage{color,xcolor,ucs}
%\usepackage{marvosym} %faces
\usepackage{hyperref}
\usepackage{multirow}


\usepackage{listings}
\usepackage{color}
\definecolor{dkgreen}{rgb}{0,0.6,0}
\definecolor{gray}{rgb}{0.5,0.5,0.5}
\definecolor{mauve}{rgb}{0.58,0,0.82}
\lstset{ %
  language=R,                     % the language of the code
  basicstyle=\TINY,           % the size of the fonts that are used for the code
  numbers=left,                   % where to put the line-numbers
  numberstyle=\tiny\color{gray},  % the style that is used for the line-numbers
  stepnumber=1,                   % the step between two line-numbers. If it's 1, each line
                                  % will be numbered
  numbersep=5pt,                  % how far the line-numbers are from the code
  backgroundcolor=\color{white},  % choose the background color. You must add \usepackage{color}
  showspaces=false,               % show spaces adding particular underscores
  showstringspaces=false,         % underline spaces within strings
  showtabs=false,                 % show tabs within strings adding particular underscores
  frame=single,                   % adds a frame around the code
  rulecolor=\color{black},        % if not set, the frame-color may be changed on line-breaks within not-black text (e.g. commens (green here))
  tabsize=1,                      % sets default tabsize to 2 spaces
  captionpos=b,                   % sets the caption-position to bottom
  breaklines=true,                % sets automatic line breaking
  breakatwhitespace=false,        % sets if automatic breaks should only happen at whitespace
  title=\lstname,                 % show the filename of files included with \lstinputlisting;
                                  % also try caption instead of title
  keywordstyle=\color{blue},      % keyword style
  commentstyle=\color{dkgreen},   % comment style
  stringstyle=\color{mauve},      % string literal style
  escapeinside={\%*}{*)},         % if you want to add a comment within your code
  morekeywords={*,...}            % if you want to add more keywords to the set
} 

% % % % % % % % % % % % % % %
%           PACKAGE CUSTOMIZATION
% % % % % % % % % % % % % % %

% GENERAL CUSTOMIZATION
\usepackage[math]{iwona}% font
\usetheme{Singapore}  % template I should use
%\usetheme{Szeged}  % alternative template
\usecolortheme{rose}  % color template
\makeatletter     % to show subsection/section title (1/3)
\beamer@theme@subsectiontrue % to show subsection/section title (2/3)
\makeatother      % to show subsection/section title (3/3)



% THIS BELOW IS TO MAKE NAVIGATION DOTS MARKED DURING PRESENTATION
\makeatletter
\def\slideentry#1#2#3#4#5#6{%
  %section number, subsection number, slide number, first/last frame, page number, part number
  \ifnum#6=\c@part\ifnum#2>0\ifnum#3>0%
    \ifbeamer@compress%
      \advance\beamer@xpos by1\relax%
    \else%
      \beamer@xpos=#3\relax%
      \beamer@ypos=#2\relax%
    \fi%
  \hbox to 0pt{%
    \beamer@tempdim=-\beamer@vboxoffset%
    \advance\beamer@tempdim by-\beamer@boxsize%
    \multiply\beamer@tempdim by\beamer@ypos%
    \advance\beamer@tempdim by -.05cm%
    \raise\beamer@tempdim\hbox{%
      \beamer@tempdim=\beamer@boxsize%
      \multiply\beamer@tempdim by\beamer@xpos%
      \advance\beamer@tempdim by -\beamer@boxsize%
      \advance\beamer@tempdim by 1pt%
      \kern\beamer@tempdim
      \global\beamer@section@min@dim\beamer@tempdim
      \hbox{\beamer@link(#4){%
          \usebeamerfont{mini frame}%
          \ifnum\c@section>#1%
            %\usebeamercolor[fg]{mini frame}%
            %\usebeamertemplate{mini frame}%
            \usebeamercolor{mini frame}%
            \usebeamertemplate{mini frame in other subsection}%
          \else%
            \ifnum\c@section=#1%
              \ifnum\c@subsection>#2%
                \usebeamercolor[fg]{mini frame}%
                \usebeamertemplate{mini frame}%
              \else%
                \ifnum\c@subsection=#2%
                  \usebeamercolor[fg]{mini frame}%
                  \ifnum\c@subsectionslide<#3%
                    \usebeamertemplate{mini frame in current subsection}%
                  \else%
                    \usebeamertemplate{mini frame}%
                  \fi%
                \else%
                  \usebeamercolor{mini frame}%
                  \usebeamertemplate{mini frame in other subsection}%
                \fi%
              \fi%
            \else%
              \usebeamercolor{mini frame}%
              \usebeamertemplate{mini frame in other subsection}%
            \fi%
          \fi%
        }}}\hskip-10cm plus 1fil%
  }\fi\fi%
  \else%
  \fakeslideentry{#1}{#2}{#3}{#4}{#5}{#6}%
  \fi\ignorespaces
  }
\makeatother


% % % % % % % % % % % % % % %
%       To show the TITLE at the Bottom of each slide
% % % % % % % % % % % % % % %

\beamertemplatenavigationsymbolsempty 
\makeatletter
\setbeamertemplate{footline}
{
\leavevmode%
\hbox{%
\begin{beamercolorbox}[wd=1\paperwidth,ht=2.25ex,dp=2ex,center]{title in head/foot}%
\usebeamerfont{title in head/foot}\insertshorttitle
\end{beamercolorbox}%
\begin{beamercolorbox}[wd=1
\paperwidth,ht=2.25ex,dp=2ex,center]{date in head/foot}%
\end{beamercolorbox}}%
}
\makeatother



% to switch off navigation bullets
%% using \miniframeson or \miniframesoff
\makeatletter
\let\beamer@writeslidentry@miniframeson=\beamer@writeslidentry
\def\beamer@writeslidentry@miniframesoff{%
  \expandafter\beamer@ifempty\expandafter{\beamer@framestartpage}{}% does not happen normally
  {%else
    % removed \addtocontents commands
    \clearpage\beamer@notesactions%
  }
}
\newcommand*{\miniframeson}{\let\beamer@writeslidentry=\beamer@writeslidentry@miniframeson}
\newcommand*{\miniframesoff}{\let\beamer@writeslidentry=\beamer@writeslidentry@miniframesoff}
\makeatother

% Image full size: use 
%%\begin{frame}
  %%\fullsizegraphic{monogram.jpg}
%%\end{frame}
\newcommand<>{\fullsizegraphic}[1]{
  \begin{textblock*}{0cm}(-1cm,-3.78cm)
  \includegraphics[width=\paperwidth]{#1}
  \end{textblock*}
}


% hyperlinks
\hypersetup{colorlinks,
            urlcolor=[rgb]{0.01, 0.28, 1.0},
            linkcolor=[rgb]{0.01, 0.28, 1.0}}


%	%	%	%	%	%	%	%	%	%	%	%	%	%	%
% 					DOCUMENT ID
%	%	%	%	%	%	%	%	%	%	%	%	%	%	%

\title{Actitudes Democr\'aticas/Autoritarias de los Habitantes de la Regi\'on: Sintomatolog\'ia de la Pre-crisis}

\author[shortname]{H\'ector Bahamonde}
\institute[shortinst]{Profesor Asistente, Universidad de O$'$Higgins}
\date{5 Dic 2019}

%to to see shadows of previous blocks
%\setbeamercovered{dynamic}
\IfFileExists{upquote.sty}{\usepackage{upquote}}{}
\begin{document}






%	%	%	%	%	%	%	%	%	%	%	%	%	%	%
% 					CONTENT
%	%	%	%	%	%	%	%	%	%	%	%	%	%	%

%% title frame
\begin{frame}
\titlepage
\end{frame}


\section{Motivaci\'on}
\subsection{Motivaci\'on}

\begin{frame}{Motivaci\'on}

  \begin{itemize}
    \item Haber tenido datos post-crisis hubiera sido potente.
    \item Sin embargo, tener datos pre-crisis podr\'ia dar se\~nales de las posibles causas de la crisis.
    \item En esta presentaci\'on pensaremos sobre ciertas actitudes de los habitantes de la regi\'on (HDLR): ¿son los HDLR m\'as {\bf democr\'aticos} o m\'as {\bf autoritarios}?
    \item Presentar\'e algunas asociaciones estad\'isticas entre estas actitudes, y otras variables.
    \item Esto podr\'ia dar ciertas se\~nales de una sintomatolog\'ia pre-crisis.
  \end{itemize}
\end{frame}


\begin{frame}{¿Por qu\'e es importante pensar el nivel de apoyo a la democracia? ¿Y qu\'e tiene que ver con la crisis?}

  \begin{itemize}
    \item En teor\'ia, las democracias debieran hacer muchas cosas. Ej., distribuir el ingreso.\\
    {\tiny Soci\'ologos pol\'iticos: ``Las elecciones son la expresi\'on democr\'atica de la lucha de clases''.}
    \item A trav\'es de elecciones, votantes debieran escoger candidatos que (entre otras cosas) prometan la re-distribuci\'on del ingreso (a trav\'es de pol\'iticas fiscales y otras pol\'iticas p\'ublicas).
    \item En esta presentaci\'on, veremos que el sistema pol\'itico ha fracasado completamente en esta labor, {\bf se\~nalando un prolongado malestar pol\'itico y social}.
  \end{itemize}
\end{frame}

\section{Metodolog\'ia}


\subsection{Datos}

\begin{frame}{Metodolog\'ia}

  \begin{itemize}
    \item Usando los mismos datos, estim\'e un modelo log\'istico multinomial (para variables categ\'oricas sin orden). La {\bf variable dependiente} es si el \emph{respondente} tiene una tendencia \emph{autoritaria}, \emph{democratica}, o es \emph{indiferente} al tipo de r\'egimen.
    \item Se ocuparon t\'ecnicas de simulaci\'on estad\'istica.
    \item Esto lo cruc\'e con distintos factores.
    %\item Formula:
  \end{itemize}

  %\begin{equation*}
  %Y_{i} = \text{Multinomial}(y_{i}|\pi_{ij})
  %\end{equation*}

 % donde,

 % \begin{equation*}
 % \pi_{ij} = \frac{\text{exp}(x_{i}\beta_{j})}{\sum_{k=1}^{J}\text{exp}(x_{i}\beta_{k})}
 % \end{equation*}

\end{frame}



\subsection{Variable Dependiente}

\begin{frame}{¿Qu\'e factores explican este comportamiento?}
    %\begin{minipage}{.7\linewidth}
        \centering
        \includegraphics[width=1\linewidth]{/Users/hectorbahamonde/RU/research/Observatorio_Regional/vd.pdf}
    %\end{minipage}
\end{frame}



\subsection{T\'ecnico}

\begin{frame}{Modelo Log\'istico Multinomial}

  \begin{equation*}
  \begin{aligned}
  %\pi_{ij} & = \frac{\text{exp}(x_{i}\beta_{j})}{\sum_{k=1}^{J}\text{exp}(x_{i}\beta_{k})}\\
  \pmb{\pi}_{ij} & = \pmb{\beta}_{j} \cdot \pmb{x}_{i}
  \end{aligned}
\end{equation*}
  
\begin{itemize}
  \item donde $\pi_{ij}$ es la probabilidad de que el sujeto $i$ tenga una preferencia de r\'egimen pol\'itico $j$,
  \item y donde $\pmb{x}$ = 
    \begin{enumerate}
        \item tendencia a cambiarse de regi\'on.
        \item confianza inter-personal.
        \item aceptaci\'on del conflicto.
        \item edad.
        \item grupo socio-econ\'omico.
        \item preferencias hacia la redistribuci\'on econ\'omica.
    \end{enumerate}
\end{itemize}

\end{frame}


\section{Pol\'itica y Econom\'ia}

\subsection{Resultados estad\'isticos}

% Aporte de los Partidos a la Regi\'on
\begin{frame}{Aporte de los Partidos a la Regi\'on}
    %\begin{minipage}{.7\linewidth}
        \centering
        \includegraphics[width=0.98\linewidth]{/Users/hectorbahamonde/RU/research/Observatorio_Regional/aporte_partidos.pdf}
    %\end{minipage}
\end{frame}


\subsection{Interpretaci\'on}


\begin{frame}{Aporte de los Partidos a la Regi\'on}
  \begin{itemize}
    \item El {\bf bajo apoyo de los partidos} a la regi\'on est\'a  sistem\'aticamente a la {\bf indiferencia} y a actitudes {\bf democr\'aticas}.
    \item Preocupantemente, el {\bf alto apoyo a los partidos}, est\'a relacionado a actitudes {\bf autoritarias}.
  \end{itemize}

Los HDLR: 
  \begin{enumerate}
  	\item quieren a los partidos lejos de la pol\'itica: {\color{red}peligro populista}.
  	\item creen que la democracia es preferible cuando los partidos pol\'iticos aportan poco a la regi\'on.
  	\item no asocian partidos pol\'iticos con democracia.
  \end{enumerate}
\end{frame}


% Grupo Socioecon\'omico

\subsection{Resultados estad\'isticos}


\begin{frame}{Grupo Socioecon\'omico}
    %\begin{minipage}{.7\linewidth}
        \centering
        \includegraphics[width=0.98\linewidth]{/Users/hectorbahamonde/RU/research/Observatorio_Regional/gse.pdf}
    %\end{minipage}
\end{frame}

\subsection{Interpretaci\'on}

\begin{frame}{Grupo Socioecon\'omico}
  \begin{itemize}
    \item Mayores ingresos correlacionan positivamente con actitudes democr\'aticas.
    \item Menores ingresos te hacen ser indiferente (democracia/dictadura: da igual).
  \end{itemize}

Los HDLR: 
  \begin{enumerate}
  	\item de menores ingresos no asocian a la democracia con re-distribuci\'on de ingresos.
  \end{enumerate}
  
  \begin{itemize}
  	\item {\bf Identificamos este factor como una de las {\color{red}principales causas} de la {\color{red}fatiga estructural del sistema}}.
  \end{itemize}

\end{frame}




% Redistribuci\'on

\subsection{Resultados estad\'isticos}

\begin{frame}{Redistribuci\'on}
    %\begin{minipage}{.7\linewidth}
        \centering
        \includegraphics[width=0.98\linewidth]{/Users/hectorbahamonde/RU/research/Observatorio_Regional/redistribucion.pdf}
    %\end{minipage}
\end{frame}

\subsection{Interpretaci\'on}

\begin{frame}{Redistribuci\'on}
  \begin{itemize}
    \item De hecho, podemos confirmar que tendencias autoritarias, democr\'aticas o indiferentes, no \emph{separan} a los HDLR, al momento de cruzar estas actitudes con preferencias hacia la redistribuci\'on de ingresos.
  \end{itemize}

Los HDLR: 
  \begin{enumerate}
  	\item no creen que la pol\'itica sea \'util para redistribuir ingresos.
  	\item podr\'ian creer (y sentirse plenamente c\'omodos) que la distribuci\'on pueda tomar bajo cualquier tipo de r\'egimen pol\'itico (incluido una dictadura).
  \end{enumerate}

  \begin{itemize}
  	\item {\bf M\'as evidencia para creer que el {\color{red}sistema pol\'itico fracas\'o} en su {\color{red}labor redistributiva}, lo que llev\'o al {\color{red}colapso infraestructural de octubre}}.
  \end{itemize}
\end{frame}


\section{Socio-Demogr\'afico}


% Edad

\subsection{Resultados estad\'isticos}

\begin{frame}{Edad}
    %\begin{minipage}{.7\linewidth}
        \centering
        \includegraphics[width=0.98\linewidth]{/Users/hectorbahamonde/RU/research/Observatorio_Regional/edad.pdf}
    %\end{minipage}
\end{frame}

\subsection{Interpretaci\'on}

\begin{frame}{Edad}
  \begin{itemize}
    \item Los j\'ovenes no valoran sistem\'aticamente m\'as la democracia, y tienden a ser m\'as indiferentes.
  \end{itemize}


Los HDLR: 
  \begin{enumerate}
  	\item j\'ovenes, los nuevos potenciales votantes, no ven la pol\'itica como algo atractivo, se\~nalando una desiluci\'on estructural con lo p\'ublico/democr\'atico.
  \end{enumerate}
\end{frame}


\section{Discusi\'on}


\subsection{Discusi\'on}

\begin{frame}{Conclusiones Preliminares: \underline{Panorama no es alentador}}

Los HDLR:
  \begin{enumerate}
    \item est\'an sufriendo una crisis estructural grave de representaci\'on pol\'itica.
    \item no ven en la democracia un catalizador v\'alido de demandas sociales, pol\'iticas y econ\'omicas.
    \item se sienten muy identificados con su regi\'on (p\'ublico), pero muy desapegados del mundo pol\'itico (p\'ublico), como es el caso de los j\'ovenes.
    %\item Los HDLR no creen que el sistema de partidos actual pueda liderar el proceso de reconfiguraci\'on socio-econ\'omica.
\end{enumerate}

\end{frame}



\end{document}
